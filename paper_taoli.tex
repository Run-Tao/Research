\documentclass[lang=cn,a4paper]{elegantpaper}

\title{基于大数据的老旧小区加装电梯出资分摊方案的数学建模研究}
\author{作者:陶理 \and 指导老师:季鑫}
\institute{上海市实验学校}
\date{\zhtoday}

\begin{document}

    \maketitle
    
    \begin{abstract}
        \keywords{老旧小区加装电梯,数学建模,大数据}
    \end{abstract}

    \section{绪论}
    \subsection{背景介绍}
    随着中国城市化进程的不断推进,各地的大规模城市建设正如火如荼地进行。与此同时,各大城市中许多上世纪90年代建造的老旧居民小区已不适应现代的生活需求,其升级改造势在必行。并且,老旧小区在各大城市中的占比很高,对于城市的美观有负面影响。所以对于老旧小区的改造势在必行。
    
    老旧小区改造中的重要一项是为低层建筑(主要为六层楼房)加装电梯。电梯的安装将极大地方便居民出行和物流服务,特别是在提升深度老龄化群体的生活品质方面具有重要的意义。
    
    在此过程中涉及到一个关键问题是居民的出资分摊方案。除政府补助部分之外,目前一般的方案是按照楼层高度以简单的正比例函数的形式进行分摊,即$C=\dfrac{kL}{N}$\footnote{$C$表示每一户需要分摊的金额;$k$表示分摊系数;$L$表示楼层数;$N$表示每层楼的户数}。然而,这一简单方案难以准确反映不同楼层对于加装电梯的实际需求以及后续相关的商业增值。一般来说,较高楼层(5、6楼)原本由于上下楼不方便价值并不显著,但是加装电梯后,高楼层采光等优势愈发明显;低楼层对于电梯的需求本来不大,而且加装电梯后采光的劣势增加。所以各楼户的增值并非与楼层成正比,原本的分摊方案并不科学经济。
    
\end{document}